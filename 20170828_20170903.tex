\documentclass{article}
\begin{document}
\title{Weekly Stock Scanner Result \\ 20170828 to 20170903}
\author{KAI YIN, CHAN}
\maketitle

\section{About the Stock Scanner}
\subsection{Model 1}
There are ten criteria included in the stock scanner. 

First of all, earnings per share (EPS) in the latest quarter should be up at least 25 \% versus the same quarter a year ago. It may imply the profitability of the company of concern improved significantly. Diluted EPS excluding extraordinary items is used.

Secondly, earning per share in current quarter should be greater than zero. This ensures the company of concern is not lossing money and increases the probability that the earning growth (criteria 1) may continue.

Thirdly. sales in the latest quarter should be up at least 25 \% versus the same quarter a year ago. While EPS may be affected by number of shares outstanding and therefore may not reflect the whole picture of profitability, sales figure can strengthen the confidence that the company is indeed profitable.

After looking at quarterly data, a year-by-year earnings increase for each of the last 3 years is required. Meaning, EPS TTM (Trailing 12 months, i.e. the past 12 months) should be greater than EPS Y1. EPS Y1 should be greater than EPS Y2. And EPS Y2 should be greater than EPS Y3. This may imply the profitability of the company does not show a downtrend.

Apart from annual earnings increase, the average annual EPS growth rate should be at least 25 \% over the last 5 years.It may imply the profitability of the company improved significantly over the years. Again diluted EPS excluding extraordinary items is used.

The next criteria is that the consensus earnings estimate for the next year should be higher than the current year. A consensus estimate is a figure based on the combined estimates of the analysts. This leverage on different analysis models to further improve our confidence that the profitability of the company is growing.

Furthermore, return on equity (ROE) should be greater than 17 \%. Compared to EPS, return on equity is a direct measure of company's profitability. It measures the return generated per unit of shareholders' equity.

Cash flow is also a concern. It is required that Cash flow per share (TTM) is greater than earnings per share (TTM) by at least 20 \%

The above 8 criteria focuses on the profitability of the company of concern. It is believed that company's profitability is correlated to stock price. Therefore, the scanner uses both annual and quarterly EPS, sales data, and ROE as well as cash flow per share to increase the prediction confidence of short term profitability growth and thus improve the likelihood that the stock price will rise.

At last, it is suggested to stand behind giant. Therefore, it is required that the company of concern should have more than 10 institutional owners, but the institutional ownership percentage should be lower than 35 \% to avoid potential price manipulation.

\subsection{Model 2}

\section{Results}
\subsection{Model 1 only}

\subsection{Model 1 and 2 combined}

\section{Actions}

\end{document}
